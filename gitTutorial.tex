\documentclass[11pt]{beamer}
\usetheme{Warsaw}
\usepackage[utf8]{inputenc}
\usepackage[english]{babel}
\usepackage{amsmath}
\usepackage{amsfonts}
\usepackage{amssymb}
\usepackage{hyperref}
\author{Thiago S. Gouvêa \\ \vspace{1cm} \url{https://github.com/tsgouvea/gitTutorial}}
\title{Using GitHub to version-control and collaborate}
%\setbeamercovered{transparent} 
%\setbeamertemplate{navigation symbols}{} 
%\logo{} 
%\institute{} 
%\date{} 
%\subject{} 
\begin{document}

\begin{frame}
\titlepage
\end{frame}

%\begin{frame}
%\tableofcontents
%\end{frame}

\begin{frame}{\textbf{git} tracks your code history}
\begin{quote}
Git is a version control system that is used for software development and other version control tasks.
\begin{flushright}
\textit{--Wikipedia}
\end{flushright}
\end{quote}
\begin{quote}
Git is a mature, actively maintained open source project originally developed in 2005 by Linus Torvalds, the famous creator of the Linux operating system kernel
\begin{flushright}
\textit{--Atlassian's tutorial}
\end{flushright}
\end{quote}
\begin{quote}
Git is a free and open source distributed version control system designed to handle everything from small to very large projects with speed and efficiency.
\begin{flushright}
\textit{--The internets somewhere}
\end{flushright}
\end{quote}
\end{frame}

\begin{frame}{Who's using it?}
\begin{description}
\item[The Brazilian Government]{\url{https://github.com/govbr}} \\ \pause
\item[Linus Torvalds]{\url{https://github.com/torvalds/linux}} \\ \pause
\item[Josh]{\url{https://github.com/sanworks}} \pause
\item[Someone mapping geographical distribution of Pokemons]{\url{https://github.com/AHAAAAAAA/PokemonGo-Map}} \pause
\item[NASA]{\url{https://github.com/nasa}} \\ \pause
\item[The Kepecs Lab]{\url{https://github.com/KepecsLab}}
\end{description}
\end{frame}

\begin{frame}{software $\neq$ GUI $\neq$ hosting service}
\begin{description}
\item[git]{the software}
\item[github/bitbucket]{hosting services}
\item[SourceTree]{the graphical user interface (GUI)}
\end{description}
\end{frame}

\begin{frame}{\textbf{git} vocabulary (part 1/2)}
\begin{description}
\item[repository]{A special folder in your computer (\textit{local repo}). \\ It is special because it is stalked by \textit{git}.\\ It is mirrored at some hosting service such as \textit{github.com} (\textit{remote repo.})} \pause
\item[clone]{A local copy of a repo that exists in a remote location.} \pause
\item[commit]{A landmark in the repo's history. \\A state of repo contents you judge worth saving. \\ You might create several a day, and push them often to the remote repo.} \pause
\item[push]{\textit{Upload} in git lingo.} \pause
\item[pull]{\textit{Download} in git lingo.}
\end{description}
\end{frame}

\begin{frame}{Hands On!}
\begin{itemize}
\item Create a git account
\item Download and install SourceTree (or go command line!)
\item Create a 'Hello World!' repo
\item Clone your repo
\item Modify, commit, push
\end{itemize}
\end{frame}

\begin{frame}{\url{github.com} tools for project management}
\begin{itemize}
\item Organizations
\item Teams
\item Issues
\item Wiki
\end{itemize}
\end{frame}

\begin{frame}{\textbf{git} vocabulary (part 2/2)}
\begin{description}
\item[tag]{A way to easily identify/locate commits of interest in the haystack} \pause
\item[pull request]{A gentler, more sociable way of uploading. \\ You don't upload (push), you request that someone downloads (pulls) from you.} \pause
\item[branch]{A parallel history thread within a repo} \pause
\item[checkout]{Move between branches} \pause
\item[detached head]{Creating new content at an ectopic site. \\ You should always be at the top of a branch (even if it's a new one)}
\item[fork]{A separate copy of a repo. \\ Not to be confused with branches.} \pause
\end{description}
\end{frame}

\begin{frame}{Tips}
\begin{itemize}
\item Keep branches simple
\item Master should always be ready for deployment
\item Pull requests are debates
\item Internet is your friend (specially stackoverflow)
\end{itemize}

\end{frame}

\begin{frame}{Further reading}
\begin{description}
\item[\textbf{Git} official documentation]{\url{https://git-scm.com/docs/gittutorial}}
\item[Atlassian's \textbf{git} tutorial]{\url{https://www.atlassian.com/git/tutorials/}}
\item[Atlassian's \textbf{SourceTree}/BitBucket tutorial]{\url{https://confluence.atlassian.com/bitbucket/tutorial-learn-sourcetree-with-bitbucket-cloud-760120235.html}}
\item[\textbf{GitFlow}: a tried-and-tested strategy for using branches]{\url{http://nvie.com/posts/a-successful-git-branching-model/}}
\item[How github uses github to build github]{\url{https://youtu.be/qyz3jkOBbQY}}
\end{description}
\end{frame}

\end{document}